%%%%%%%%%%%%%%%%%%%%%%%%%%%%%%%%%%%%%%%%%
% University/School Laboratory Report
% LaTeX Template
% Version 3.1 (25/3/14)
%
% This template has been downloaded from:
% http://www.LaTeXTemplates.com
%
% Original author:
% Linux and Unix Users Group at Virginia Tech Wiki 
% (https://vtluug.org/wiki/Example_LaTeX_chem_lab_report)
%
% License:
% CC BY-NC-SA 3.0 (http://creativecommons.org/licenses/by-nc-sa/3.0/)
%
%%%%%%%%%%%%%%%%%%%%%%%%%%%%%%%%%%%%%%%%%

%----------------------------------------------------------------------------------------
%	PACKAGES AND DOCUMENT CONFIGURATIONS
%----------------------------------------------------------------------------------------

\documentclass{article}
\newcommand{\shellcmd}[1]{\\\indent\indent\texttt{\footnotesize\# #1}\\}
\usepackage{graphicx} % Required for the inclusion of images
\graphicspath{ {images/} }
\usepackage{natbib} % Required to change bibliography style to APA
\usepackage{amsmath} % Required for some math elements 

\setlength\parindent{0pt} % Removes all indentation from paragraphs


%\usepackage{times} % Uncomment to use the Times New Roman font

%----------------------------------------------------------------------------------------
%	DOCUMENT INFORMATION
%----------------------------------------------------------------------------------------

\title{Project 1: Performance Analysis of IPC Paradigms \\ NWEN 401} % Title

\author{Sriram Venkatesh \\ Victoria University of Wellington} % Author name

\date{\today} % Date for the report

\begin{document}

\maketitle % Insert the title, author and date



% If you wish to include an abstract, uncomment the lines below
% \begin{abstract}
% Abstract text
% \end{abstract}

%----------------------------------------------------------------------------------------
%	SECTION 1
%----------------------------------------------------------------------------------------

\section{Introduction}
The goal of the project was to evaluate the performance of the following three inter-process communication mechanisms:

% If you have more than one objective, uncomment the below:
\begin{enumerate}
	\item Java Sockets
	\item Java Remote Method Invocation (Java RMI)
	\item Java Message Service using JBoss Package
\end{enumerate}

\subsection{Description}


\subsection{Aim}
Although these mechanisms are principally different protocols, there are always certain situations where remote applications can be implemented using either of these protocols. In such situations, programmers need to choose one of them based on some reliable and concrete results of performance comparision of these two protocols. \\


\subsection{Problems to be Solved}
After the completion of the experiment, the following questions will be answered:
\begin{itemize}
	\item What is the relative performance of each IPC method?
	\item How does the features of each method explain the difference in performance?	
	\item How does the size of the message being sent affect the performance of the protocol?
\end{itemize}

\subsection{Performance Comparison}
In this project report, I have conducted the perofmrance comparison of Java Socket Via TCP, Java RMI and JMS based on three main factors:
\begin{itemize}
\item Latency
\item Throughput
\end{itemize}

\subsection{Experiment Scenarios}
Each IPC was tested under the following conditions:
\begin{itemize}
	\item Each client will send 50 requests to the server.
	\item The client and server were run the same local network, and the client and server were run
	on different network.
	\item Tests were all run using a Unix Machine running similar specs. 
	\item For testing under different networks, a SSH Tunnel was created between the remote and local machine.
\end{itemize}

\subsection{Performance Criteria}
For the purposes of this project we have measured the round trip times (RTT) and throughput of each application. The RTT is the total time a method invocation takes. It is important to not that the methods used for perfomrance evaluation do not do any processing and therefore the RTT expresses the overheard of the method invocation. Throughput was calculated by dividing total received packet size by duration \\

In the each of the test cases, the network is fast and has a high bandwidth.





 
%----------------------------------------------------------------------------------------
%	SECTION 2
%----------------------------------------------------------------------------------------

\section{Apparatus}

\subsection{Software Used}

\subsection{Experimental Environment}
\begin{table}
    \begin{tabular}{|c|c|}
    \hline
    Parameter        & Value                                    \\ \hline
    Operating System & elementary OS Luna     					\\ \hline
    Java Version     & OpenJDK v1.7.0\_51                       \\ \hline
    CPU              & Intel(R) Core(TM) i7-2620M CPU @ 2.70GHz \\ \hline
    RAM              & 8GB                                      \\ \hline
    \end{tabular}
    \caption {Client Computer Specifications}
\end{table}


\begin{table}
    \begin{tabular}{|c|c|}
    \hline
    Parameter        & Value                                    \\ \hline
    Operating System & Ubuntu 12.04.3 LTS    					\\ \hline
    Java Version     & OpenJDK v1.7.0\_51                       \\ \hline
    CPU              & Intel(R) Pentium(R) 4 CPU 2.80GHz 		\\ \hline
    RAM              & 1GB                                      \\ \hline
    \end{tabular}
    \caption {Server Computer Specifications}
\end{table}

%----------------------------------------------------------------------------------------
%	SECTION 3
%----------------------------------------------------------------------------------------

\section{Procedures}

\subsection{Methodology}
Developed a simple client server application in Java. The Client sends 100000 messages with the varying sizes. 

The sizes of the message are displayed below: 
\begin{itemize}
	\item 10 kilobytes
	\item 15 kilobytes
	\item 20 kilobytes
	\item 25 kilobytes
	\item 30 kilobytes
	\item 35 kilobytes
\end{itemize}

To make sure my tests were reliable we ran each test 5 times and gathered statitcs based on the results.



%----------------------------------------------------------------------------------------
%	SECTION 4
%----------------------------------------------------------------------------------------

\section{Performance Results}





%----------------------------------------------------------------------------------------
%	SECTION 5
%----------------------------------------------------------------------------------------

\section{Discussion}


%----------------------------------------------------------------------------------------
%	SECTION 6
%----------------------------------------------------------------------------------------

\section{Conclusions and recommendations}

Performance has an important influence on the usability of distributed applications. Therefore it is useful to know, how well do these distributed models perform. In this project we have clarifed this question. We have compared the relative performance of RMI, JMS and Java Sockets for the Java 7 verion. We have measured the performance on a Linux Computer.

What is the relative performance of each IPC method?\\


How does the features of each method explain the difference in performance? \\


Does the location (Local Network/Different Network) of the client/server pair affect the performance of the IPC? \\



\subsection{Future Extensions}
Future projects should compare the relative performance of using these methods of IPC on different operating systems and validate a wheather the operating system affects the performance of these distributed methods. 

Another comparision should be made between different transport protocols such as UDP and TCP to see which performs better in different circumstances.


%----------------------------------------------------------------------------------------
%	BIBLIOGRAPHY
%----------------------------------------------------------------------------------------

\bibliographystyle{apalike}

\bibliography{sample}

%----------------------------------------------------------------------------------------


\end{document}